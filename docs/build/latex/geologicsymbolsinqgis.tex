%% Generated by Sphinx.
\def\sphinxdocclass{report}
\documentclass[letterpaper,10pt,english]{sphinxmanual}
\ifdefined\pdfpxdimen
   \let\sphinxpxdimen\pdfpxdimen\else\newdimen\sphinxpxdimen
\fi \sphinxpxdimen=.75bp\relax

\PassOptionsToPackage{warn}{textcomp}
\usepackage[utf8]{inputenc}
\ifdefined\DeclareUnicodeCharacter
% support both utf8 and utf8x syntaxes
  \ifdefined\DeclareUnicodeCharacterAsOptional
    \def\sphinxDUC#1{\DeclareUnicodeCharacter{"#1}}
  \else
    \let\sphinxDUC\DeclareUnicodeCharacter
  \fi
  \sphinxDUC{00A0}{\nobreakspace}
  \sphinxDUC{2500}{\sphinxunichar{2500}}
  \sphinxDUC{2502}{\sphinxunichar{2502}}
  \sphinxDUC{2514}{\sphinxunichar{2514}}
  \sphinxDUC{251C}{\sphinxunichar{251C}}
  \sphinxDUC{2572}{\textbackslash}
\fi
\usepackage{cmap}
\usepackage[T1]{fontenc}
\usepackage{amsmath,amssymb,amstext}
\usepackage{babel}



\usepackage{times}
\expandafter\ifx\csname T@LGR\endcsname\relax
\else
% LGR was declared as font encoding
  \substitutefont{LGR}{\rmdefault}{cmr}
  \substitutefont{LGR}{\sfdefault}{cmss}
  \substitutefont{LGR}{\ttdefault}{cmtt}
\fi
\expandafter\ifx\csname T@X2\endcsname\relax
  \expandafter\ifx\csname T@T2A\endcsname\relax
  \else
  % T2A was declared as font encoding
    \substitutefont{T2A}{\rmdefault}{cmr}
    \substitutefont{T2A}{\sfdefault}{cmss}
    \substitutefont{T2A}{\ttdefault}{cmtt}
  \fi
\else
% X2 was declared as font encoding
  \substitutefont{X2}{\rmdefault}{cmr}
  \substitutefont{X2}{\sfdefault}{cmss}
  \substitutefont{X2}{\ttdefault}{cmtt}
\fi


\usepackage[Bjarne]{fncychap}
\usepackage{sphinx}

\fvset{fontsize=\small}
\usepackage{geometry}

% Include hyperref last.
\usepackage{hyperref}
% Fix anchor placement for figures with captions.
\usepackage{hypcap}% it must be loaded after hyperref.
% Set up styles of URL: it should be placed after hyperref.
\urlstyle{same}
\addto\captionsenglish{\renewcommand{\contentsname}{Contents:}}

\usepackage{sphinxmessages}
\setcounter{tocdepth}{1}



\title{Geologic symbols in QGIS}
\date{Jun 25, 2019}
\release{1.0}
\author{Alessandro Frigeri}
\newcommand{\sphinxlogo}{\vbox{}}
\renewcommand{\releasename}{Release}
\makeindex
\begin{document}

\pagestyle{empty}
\sphinxmaketitle
\pagestyle{plain}
\sphinxtableofcontents
\pagestyle{normal}
\phantomsection\label{\detokenize{index::doc}}



\chapter{Introduction}
\label{\detokenize{intro:introduction}}\label{\detokenize{intro::doc}}
Geologic mapping is an interpretative process which usually is finalized in the production of a geologic map, where a wide range of symbols are used to visually encode the information over the map sheet.

Symbology is divided into three main categories:
\begin{itemize}
\item {} 
line symbols: describing linear geologic feature, from geologic contact to faults and lineations

\item {} 
point symbols: describing point measurements (strike, dip) or describing features too small to be mapped differently

\item {} 
area patterns: differentiating one terrain/material/unit from another

\end{itemize}


\section{Guidelines}
\label{\detokenize{intro:guidelines}}

\chapter{Installation}
\label{\detokenize{installation:installation}}\label{\detokenize{installation::doc}}
To install the geologic symbol library you need to download the zipfile of the current distribution, which is available at the github project’s page: \sphinxurl{http://www.github.com/afrigeri/geologic-symbols-qgis}.

Once you have the geologic\_symblib.zip, decompress it in a directory of your choice.

For example, if you are running a Mac OSX, you can extract the contents of the zipfile into a folder called \_\_foo\_\_.

Within the \_\_foo\_\_ directory you will find:
\begin{enumerate}
\def\theenumi{\arabic{enumi}}
\def\labelenumi{\theenumi .}
\makeatletter\def\p@enumii{\p@enumi \theenumi .}\makeatother
\item {} 
an xml file called geologic\_symblib.xml, which is the symbol library

\item {} 
a folder called svg, which contains svg graphics and patterns

\end{enumerate}

Now we can configure QGIS to use the symbol library.

First we tell QGIS where to look for SVG pattern and graphics.  In the main program go to ‘Settings -\textgreater{} Options’, if you have Linux, or ‘Preferences’ if you run on OSX.  From the panel, select the second tab from the top: ‘System’.

In the ‘SVG Paths’ form, click on the ‘+’ button and select the \_\_foo\_\_ directory where the svg folder  and xml file are located.

Then got to the main QGIS menu and open the ‘Style Manager’ window from the ‘Settings’ menu.

At the bottom left of the ‘Style Manager’ click on ‘Import/Export’ button and then ‘Import items’.  Now select the geologic\_symblib.xml file.

You should now have the geologic symbol library available in QGIS.  From the Style Manager you can select all the symbols or only a sub-group of them.  For example, you can select only the FGDC symbols by clicking on the fgdc tag on the left part of the Style Manager.


\chapter{Usage and conventions}
\label{\detokenize{usage:usage-and-conventions}}\label{\detokenize{usage::doc}}

\chapter{Adding new symbols to the library}
\label{\detokenize{developers_guide:adding-new-symbols-to-the-library}}\label{\detokenize{developers_guide::doc}}

\section{Guidelines for new symbols}
\label{\detokenize{developers_guide:guidelines-for-new-symbols}}\begin{enumerate}
\def\theenumi{\arabic{enumi}}
\def\labelenumi{\theenumi .}
\makeatletter\def\p@enumii{\p@enumi \theenumi .}\makeatother
\item {} 
The symbol directory corresponds with the realative authority

\item {} 
The symbol’s name is composed by two parts: {[}NAME{]} : {[}Description{]}

\end{enumerate}


\chapter{Symbol’s developer/designer guide}
\label{\detokenize{developers_guide:symbol-s-developer-designer-guide}}
This guide will drive you through the process of designing new symbols you can submit to the main repository.

All you have to do is to design your symbol and export it to an xml file which can then be added to the current library.


\section{The Style Manager}
\label{\detokenize{developers_guide:the-style-manager}}
The style manager allow to design a symbol from scratch.

\noindent\sphinxincludegraphics{{style_manager}.png}

\# Authority’s specific Conventions

\#\#\# FGCD

Naming:
\begin{quote}

XX.YYY: string
\end{quote}

here XX is the name of the FGDC section, and YYY is the code of the symbol, padded with 0s if the number is lower than 100.

This is the tutorial


\chapter{Indices and tables}
\label{\detokenize{index:indices-and-tables}}\begin{itemize}
\item {} 
\DUrole{xref,std,std-ref}{genindex}

\item {} 
\DUrole{xref,std,std-ref}{modindex}

\item {} 
\DUrole{xref,std,std-ref}{search}

\end{itemize}



\renewcommand{\indexname}{Index}
\printindex
\end{document}